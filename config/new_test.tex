% docx2tex 1.1 --- ``Escape from MS Island''
%
% docx2tex is Open Source and
% you can download it on GitHub:
% https://github.com/transpect/docx2tex
%
\documentclass{scrbook}
\usepackage{graphicx}
\usepackage{hyperref}
\usepackage{multirow}
\usepackage{tabularx}
\usepackage{color}
\usepackage{amsmath}
\usepackage{amssymb}
\usepackage{amsfonts}
\usepackage{amsxtra}
\usepackage{wasysym}
\usepackage{isomath}
\usepackage{mathtools}
\usepackage{txfonts}
\usepackage{upgreek}
\usepackage{enumerate}
\usepackage{tensor}
\usepackage{pifont}

\begin{document}
贵阳市普通高中 2018 届高三年级 8 月摸底考试

理科数学参考答案

@P@一、选择题 本大题共 12 小题,每小题 5 分,共 60 分.在每小题给出的四个选项中,只有一项是符合题目要求的.

@100MA{\textbar}O1@1

@答案@ B

@典型错误@

@解析@

@100MA{\textbar}O2@2

@答案@C

@典型错误@

@解析@

@100MA{\textbar}O3@3

@答案@C

@典型错误@

@解析@

@100MA{\textbar}O4@4

@答案@A

@典型错误@

@解析@

@100MA{\textbar}O5@5

@答案@D

@典型错误@

@解析@

@100MA{\textbar}O6@6

@答案@D

@典型错误@

@解析@

@100MA{\textbar}O7@7

@答案@B

@典型错误@

@解析@

@100MA{\textbar}O8@8

@答案@B

@典型错误@

@解析@

@100MA{\textbar}O9@9

@答案@B

@典型错误@

@解析@

@100MA{\textbar}O10@10

@答案@D

@典型错误@

@解析@

@100MA{\textbar}O11@11

@答案@A

@典型错误@

@解析@

@100MA{\textbar}O12@12

@答案@A

@典型错误@

@解析@

@P@二.填空题:本大题共 4 小题,每小题 5 分,共 20 分.把各题答案的最简形式写在题中的横线上.

@100MA{\textbar}T13@13

@答案@-3

@典型错误@

@解析@



@100MA{\textbar}T14@14

@答案@4

@典型错误@

@解析@



@100MA{\textbar}T15@15

@答案@ 0

@典型错误@

@解析@

@100MA{\textbar}T16@16

@答案@$n\cdot 2^{n+1}$

@典型错误@

@解析@

@P@三、解答题 本大题共 70 分.解答应写出文字说明、证明过程或演算步骤.

@110MA{\textbar}S17@17

@答案@

@典型错误@

@解析@

@101MA{\textbar}S17\#1@(Ⅰ)

@答案@

@典型错误@

@解析@

解:由题意得$\mathrm{b}=\mathrm{a}+2,\mathrm{c}=\mathrm{a}+4$,由余弦定理$\cos C=\frac{a^{2}+b^{2}- c^{2}}{2ab}$得

$\cos ^{120}=\frac{a^{2}+(a+2)^{2}- (a+4)^{2}}{2a(a+2)}$即$a^{2}- a- 6=0,\therefore a=3$或$\mathrm{a}=-2$(舍去)$\therefore a=3,$.......6分

@101MA{\textbar}S17\#2@(Ⅱ)

@答案@

@典型错误@

@解析@

解法1:由(Ⅰ)知$\mathrm{a}=3,\mathrm{b}=5,\mathrm{c}=7$,由三角形的面积公式得

$\frac{1}{2}ab\sin C=\frac{1}{2}c\times CD$ $\therefore CD=\frac{ab\sin C}{c}=\frac{3\times 5\times \frac{\sqrt{3}}{2}}{7}=\frac{15\sqrt{3}}{14}$

即 AB 边上的高$CD=\frac{15\sqrt{3}}{14}$.......12分

解法 2:由(Ⅰ)知$\mathrm{a}=3,\mathrm{b}=5,\mathrm{c}=7$

由正弦定理得$\frac{3}{\sin A}=\frac{7}{\sin C}=\frac{7}{\sin ^{120}}$即$\sin A=\frac{3\sqrt{3}}{14}$

在$Rt\Updelta ACD$中$CD=AC\sin A=5\times \frac{3\sqrt{3}}{14}=\frac{15\sqrt{3}}{14}$

即AB边上的高$CD=\frac{15\sqrt{3}}{14}$\raisebox{-12pt}{{\ldots}{\ldots}}\raisebox{-12pt}{12分}

@110MA{\textbar}S18@18

@答案@

@典型错误@

@解析@

@101MA{\textbar}S18\#1@(Ⅰ)

@答案@

@典型错误@

@解析@

解:男生打的平均分为

\begin{equation*}
\frac{1}{10}(55+53+62+65+71+70+73+74+86+81)=69
\end{equation*}

由茎叶图知,女生打分比较集中,男生打分比较分散;{\ldots}{\ldots}6分

@101MA{\textbar}S18\#2@(Ⅱ)

@答案@

@典型错误@

@解析@

因为打分在 80 分以上的有 3 女 2 男,

${\therefore}$ X 的可能取值为1,2,3

\begin{equation*}
P(X=1)=\frac{C_{3}^{1}C_{2}^{2}}{C_{5}^{3}}=\frac{3}{10},P(X=2)=\frac{C_{3}^{2}C_{2}^{1}}{C_{5}^{3}}=\frac{3}{5},P(X=3)=\frac{C_{3}^{3}C_{2}^{0}}{C_{5}^{3}}=\frac{1}{10}
\end{equation*}

\raisebox{0.5pt}{${\therefore}$} \textit{X} \raisebox{0.5pt}{的分布列为}

\includegraphics[width=1\textwidth]{http://res.eval.jyjy.cn/FunDp4x7fb1TxjAhAWlKd7lYJXQP}

$E(X)=1\times \frac{3}{10}+2\times \frac{3}{5}+3\times \frac{1}{10}=\frac{9}{5}$......12分

@110MA{\textbar}S19@19.

@答案@

@典型错误@

@解析@

@101MA{\textbar}S19\#1@(Ⅰ)

@答案@

@典型错误@

@解析@

证明:由圆柱性质知,$DA\bot \text{平面ABE}$ ,

又$BE\subset \text{平面ABE}$ ,$\therefore BE\bot DA$

又 \textit{AB} 是底面圆的直径, \textit{E} 是底面圆周上不同于 \textit{A}, \textit{B} 两点的一点,$\therefore BE\bot AE$

又$DA\cap AE=A$,$DA,AE\subset \mathrm{平面}DAE$,

${\therefore}$ $\therefore BE\bot \mathrm{平面}DAE$ 6 分

@101MA{\textbar}S19\#2@(Ⅱ)

@答案@

@典型错误@

@解析@

解法 1:

过$E$作$EF\bot AB$,垂足为$F$,由圆柱性质知$\mathrm{平面}ABCD\bot \mathrm{平面}ABE$,

 \includegraphics[width=1\textwidth]{http://res.eval.jyjy.cn/Fqzvx41bN48Ik-5GvoWvbBcQ8i1L}

\raisebox{0.5pt}{${\therefore}$} $EF\bot \mathrm{平面}ABCD$,又过 $F$作 $FH\bot DB$ ,垂足为 \textit{\raisebox{0.5pt}{H}} ,连接 \textit{\raisebox{0.5pt}{EH}} , 则$\angle EHF$即为所求的二面角的平面角的补角,$AB=AD=2,AE=1$易得$DE=\sqrt{5},BE=\sqrt{3},BD=2\sqrt{2}$

\begin{equation*}
\therefore EF=\frac{AE\times BE}{AB}=\frac{\sqrt{3}}{2}
\end{equation*}

由(Ⅰ)知$BE\bot DE,\therefore EH=\frac{DE\times BE}{DB}=\frac{\sqrt{5}\times \sqrt{3}}{2\sqrt{2}}=\frac{\sqrt{30}}{4}$

$\therefore \sin \angle EHF=\frac{EF}{EH}=\frac{\frac{\sqrt{3}}{2}}{\frac{\sqrt{30}}{4}}=\frac{\sqrt{10}}{5},\therefore \cos \angle EHF=\sqrt{1- \sin ^{2}\angle EHF}=\frac{\sqrt{15}}{5}$${\therefore}$所求的二面角的余弦值为$- \frac{\sqrt{15}}{5}${\ldots}{\ldots}12 分

解法 2:过 \textit{\raisebox{0.5pt}{A}} 在平面$\mathrm{AEB}$\raisebox{-1.5pt}{作}$Ax\bot AB$\raisebox{-1.5pt}{,}建立如图的空间直角坐标系,

\includegraphics[width=1\textwidth]{http://res.eval.jyjy.cn/FjYX0Tvg8m_j4LXEimXK3HZ7FoL-}

$\because AB=AD=2,AE=1,\therefore BE=\sqrt{3},\therefore E\left(\frac{\sqrt{3}}{2},\frac{1}{2},0\right),D(0,0,2),B(0,2,0)$$\therefore \overset{\rightarrow }{ED}=\left(- \frac{\sqrt{3}}{2},- \frac{1}{2},2\right),\overset{\rightarrow }{BD}=(0,- 2,2)$

平面$\mathrm{CDB}$的法向量为$\mathbf{n}_{1}=(1,0,0)$,设平面$\mathrm{EBD}$的法向量为$\mathbf{n}_{2}=\left(x_{2},y_{2},z_{2}\right)$

\begin{align*}
\left\{\begin{array}{l}
\mathbf{n}_{2}\cdot \overset{\rightarrow }{ED}=0\\
\mathbf{n}_{2}\cdot \overset{\rightarrow }{BD}=0
\end{array}\right.
\end{align*}

即$\left\{\begin{array}{l}
- \frac{\sqrt{3}}{2}x_{2}- \frac{1}{2}y_{2}+2z_{2}=0\\
- 2y_{2}+2z_{2}=0
\end{array}\right.$ 取$n_{2}=(\sqrt{3},1,1)$

\begin{equation*}
\therefore \cos <n_{1},n_{2}>=\frac{n_{1}\cdot n_{2}}{\left|n_{1}\right|\left|n_{2}\right|}=\frac{\sqrt{3}}{\sqrt{5}}=\frac{\sqrt{15}}{5}
\end{equation*}

${\therefore}$所求的二面角的余弦值为$- \frac{\sqrt{15}}{5}${\ldots}{\ldots}12 分

解法 3:如图,以 \textit{E} 为原点, \textit{EB}, \textit{EA} 分别为 \textit{x} 轴, \textit{y} 轴,圆柱过点 \textit{E} 的母线为 \textit{z} 轴建立空

间直角坐标系,则

\includegraphics[width=1\textwidth]{http://res.eval.jyjy.cn/Fouj3_mB0EtP3uNJSPp7jLGlq02y}

\begin{equation*}
A(0,1,0),B(\sqrt{3},0,0),C(\sqrt{3},0,2),D(0,1,2),E(0,0,0)
\end{equation*}

\begin{equation*}
\therefore \overset{\rightarrow }{BC}=(0,0,2),\overset{\rightarrow }{CD}=(- \sqrt{3},1,0),\overset{\rightarrow }{BD}=(- \sqrt{3},1,2)\overset{\rightarrow }{EB}=(\sqrt{3},0,0)
\end{equation*}

设$\mathbf{n}_{1}=(x,y,z)$是平面 \textit{BCD} 的一个法向量,

则$\mathbf{n}_{2}\bot \overset{\rightarrow }{BD},\mathbf{n}_{2}\bot \overset{\rightarrow }{EB}$,即$\left\{\begin{array}{l}
- \sqrt{3}x+y+2z=0\\
\sqrt{3}x+0y+0z=0
\end{array}\right.$令$\mathrm{z}=1$,则$\mathrm{y}=-2,\mathrm{x}=0$

\begin{equation*}
\therefore n_{2}=(0,- 2,1),\left|n_{2}\right|=\sqrt{5}
\end{equation*}

\begin{equation*}
\therefore \cos <n_{1},n_{2}>=\frac{n_{1}\cdot n_{2}}{\left|n_{1}\right|\cdot \left|n_{2}\right|}=\frac{- 2\sqrt{3}}{2\sqrt{5}}=- \frac{\sqrt{15}}{5}
\end{equation*}

${\therefore}$所求的二面角的余弦值为$- \frac{\sqrt{15}}{5}${\ldots}{\ldots}12 分

解法 4:由(Ⅰ)知可建立如图的空间直角坐标系

$\because AB=AD=2,AE=1,\therefore BE=\sqrt{3},\therefore E(0,0,0),D(1,0,2),B(0,\sqrt{3},0),C(0,\sqrt{3},2)$$\therefore \overset{\rightarrow }{ED}=(1,0,2),{\quad} \overset{\rightarrow }{EB}=(0,\sqrt{3},0),{\quad} \overset{\rightarrow }{BD}=(1,- \sqrt{3},2),{\quad} \overset{\rightarrow }{BC}=(0,0,2)$设平面\textit{CDB} 的法向量为$\mathbf{n}_{1}=\left(x_{1},y_{1},z_{1}\right)$,平面 \textit{EBD} 的法向量为$\mathbf{n}_{2}=\left(x_{2},y_{2},z_{2}\right)$

\begin{align*}
\therefore \left\{\begin{array}{l}
\mathbf{n}_{1}\cdot \overset{\rightarrow }{BD}=0\\
\mathbf{n}_{1}\cdot \overset{\rightarrow }{BC}=0
\end{array}\right.,{\quad} \left\{\begin{array}{l}
\mathbf{n}_{2}\cdot \overset{\rightarrow }{ED}=0\\
\mathbf{n}_{2}\cdot \overset{\rightarrow }{EB}=0
\end{array}\right.
\end{align*}

即$\left\{\begin{array}{l}
x_{1}- \sqrt{3}y_{1}+2z_{1}=0\\
2z_{1}=0
\end{array}\right.,{\quad} n_{1}=(\sqrt{3},1,0)$

$\left\{\begin{array}{l}
x_{2}+2z_{2}=0\\
\sqrt{3}y_{2}=0
\end{array}\right.$,取$\mathbf{n}_{2}=(- 2,0,1)$

\begin{equation*}
\therefore \cos <n_{1},n_{2}>=\frac{n_{1}\cdot n_{2}}{\left|n_{1}\right|\left|n_{2}\right|}=\frac{- 2\sqrt{3}}{2\sqrt{5}}=- \frac{\sqrt{15}}{5}
\end{equation*}

${\therefore}$所求的二面角的余弦值为$- \frac{\sqrt{15}}{5}$

@110MA{\textbar}S20@20

@答案@

@典型错误@

@解析@

@101MA{\textbar}S20\#1@ (Ⅰ)

@答案@

@典型错误@

@解析@

解: \textit{\raisebox{0.5pt}{F}} 的坐标为$(1,0)$ ,设\textit{l} 的方程为 $\mathrm{y}=\mathrm{k}(\mathrm{x}-1)$代入抛物线 $y^{2}=4x$得

\begin{equation*}
k^{2}x^{2}- \left(2k^{2}+4\right)x+k^{2}=0
\end{equation*}

由题意知$k\neq 0$,且$\left[- \left(2k^{2}+4\right)\right]^{2}- 4k^{2}\cdot k^{2}=16\left(k^{2}+1\right)>0$

设$A\left(x_{1},y_{1}\right),B\left(x_{2},y_{2}\right),\therefore x_{1}+x_{2}=\frac{2k^{2}+4}{k^{2}},x_{1}x_{2}=1$

由抛物线的定义知$|\mathrm{AB}|=\mathrm{x}_{1}+\mathrm{x}_{2}+2=8$

$\therefore \frac{2k^{2}+4}{k^{2}}=6,\therefore k^{2}=1$即$k=\pm 1$,${\therefore}$直线\textit{l}的方程为 $y=\pm (x- 1)${\ldots}{\ldots}{\ldots}{\ldots}{\ldots}{\ldots}{\ldots}6 分

@101MA{\textbar}S20\#1@(Ⅱ)

@答案@

@典型错误@

@解析@

由抛物线的对称性知, \textit{D} 点的坐标$\left(x_{1},- y_{1}\right)$

直 \textit{BD} 的斜率为$k_{BD}=\frac{y_{2}+y_{1}}{x_{2}- x_{1}}=\frac{y_{2}+y_{1}}{\frac{y_{2}}{4}- \frac{y_{1}^{2}}{4}}=\frac{4}{y_{2}- y_{1}}$

${\therefore}$直线 \textit{BD} 的方程为$y+y_{1}=\frac{4}{y_{2}- y_{1}}\left(x- x_{1}\right)$

即$\left(y_{2}- y_{1}\right)y+y_{2}y_{1}- y_{1}^{2}=4x- 4x_{1}$

\begin{equation*}
\because y^{2}=4x{\quad} ,{\quad} x_{1}x_{2}=1\therefore \left(y_{1}y_{2}\right)^{2}=16x_{1}x_{2}=16
\end{equation*}

即$\mathrm{y}_{1}\mathrm{y}_{2}=-4$(因为$\mathrm{y}_{1},\,\mathrm{y}_{2}$异号)

${\therefore}$ \textit{BD} 的方程为$4(x+1)+\left(y_{1}- y_{2}\right)y=0$,恒过$(-1,0)$

@110MA{\textbar}S21@21

@答案@

@典型错误@

@解析@

@101MA{\textbar}S21\#1@(Ⅰ)

@答案@

@典型错误@

@解析@

解:方法 1:$f(x)=kx- \ln x- 1,f^{\prime }(x)=k- \frac{1}{x}=\frac{kx- 1}{x}(x>0,k>0)$

$x=\frac{1}{k}$时,$f^{\prime }(x)=0$,$0<x<\frac{1}{k}$时,$f^{\prime }(x)<0$,$x>\frac{1}{k}\mathrm{时}$,$f^{\prime }(x)>0$

$\therefore f(x)$在$\left(0,\frac{1}{k}\right)$上单调递减,在$\left(\frac{1}{k},+\infty \right)$上单调递增;

$\therefore f(x)_{\min }=f\left(\frac{1}{k}\right)=\ln k$ $\because f(x)\text{有且只有一个零点}$

故$\ln k=0,\therefore k=1${\ldots}{\ldots}6 分

方法 2:由题意知方程$kx- \ln x- 1=0$仅有一实根,

由$kx- \ln x- 1=0$得$k=\frac{\ln x+1}{x}(x>0)$

\begin{equation*}
\mathrm{x}=1\mathrm{时},g^{\prime }(x)=0;0<x<1\mathrm{时},g^{\prime }(x)>0;x>1\mathrm{时},g^{\prime }(x)<0
\end{equation*}

\begin{equation*}
\therefore \mathrm{g}(\mathrm{x})\mathrm{在}(0,1)\text{上单调递增},\mathrm{在}(1,+\infty )\text{单调递减}
\end{equation*}

\begin{equation*}
\therefore g(x)_{\max }=g(1)=1
\end{equation*}

所以要使 $f(x)$ 仅有一个零点,则 $\mathrm{k}=1$。{\ldots}{\ldots}6 分

方法 3:函数 $f(x)$有且只有一个零点即为直线$\mathrm{y}=\mathrm{k}x$与曲线$y=\ln x+1$相切,设切点为

$\left(x_{0},y_{0}\right)$,由$y=\ln x+1$得$y^{\prime }=\frac{1}{x}$$\therefore \left\{\begin{array}{l}
k=\frac{1}{x_{0}}\\
y_{0}=kx_{0}\\
y_{0}=\ln x_{0}+1
\end{array}\right.,\therefore k=x_{0}=y_{0}=1$

所以实数\textit{k} 的值为 1.{\ldots}{\ldots}6 分

@101MA{\textbar}S21\#2@(Ⅱ)

@答案@

@典型错误@

@解析@

由(Ⅰ)知 $x- \ln x- 1\geq 0$,即$x- 1\geq \ln x$当且仅当$\mathrm{x}=1$时取等号,

\begin{equation*}
\because n\in N^{*},\mathrm{令}x=\frac{n+1}{n},\mathrm{得}\frac{1}{n}>\ln \frac{n+1}{n}
\end{equation*}

\begin{equation*}
1+\frac{1}{2}+\frac{1}{3}+\cdots +\frac{1}{n}>\ln \frac{2}{1}+\ln \frac{3}{2}+\cdots +\ln \frac{n+1}{n}=\ln (n+1)
\end{equation*}

即$1+\frac{1}{2}+\frac{1}{3}+\cdots +\frac{1}{n}>\ln (n+1)${\ldots}{\ldots}12分

@110MA{\textbar}S22@22

@答案@

@典型错误@

@解析@

@101MA{\textbar}S22\#1@\raisebox{-2.5pt}{(Ⅰ)}

@答案@

@典型错误@

@解析@

\raisebox{-2.5pt}{解: C 的直角坐标方程为}$\frac{x^{2}}{4}+y^{2}=1$

\raisebox{-2.5pt}{由}$\rho \cos \left(\theta +\frac{\pi }{4}\right)=\sqrt{2}$\raisebox{-2.5pt}{得}$x-\mathrm{y}-2=0$\raisebox{-2.5pt}{,}直线\textit{l}的倾斜角为$\frac{\pi }{4}$

过点$(2,0)$ ,故直线\textit{l}的一个参数方程为$\left\{\begin{array}{l}
x=2+\frac{\sqrt{2}}{2}t\\
y=\frac{\sqrt{2}}{2}t
\end{array}\right.$($\mathrm{t}$为参数)

{\ldots}{\ldots}5 分

@101MA{\textbar}S22\#2@(Ⅱ)

@答案@

@典型错误@

@解析@

将\textit{l} 的参数方程带入\textit{C} 的直角坐标方程得

\begin{equation*}
5t^{2}+4\sqrt{2}t=0,{\quad} t_{1}=0,t_{2}=- \frac{4\sqrt{2}}{5}
\end{equation*}

显然\textit{l} 与\textit{C} 有两个交点 \textit{A}, \textit{B} 且$|AB|=\left|t_{1}- t_{2}\right|=\frac{4\sqrt{2}}{5}${\ldots}{\ldots}10分

@110MA{\textbar}S23@23

@答案@

@典型错误@

@解析@

@101MA{\textbar}S23\#1@\raisebox{-2.5pt}{(Ⅰ)}

@答案@

@典型错误@

@解析@

解:$f(x)\geq 6,\mathrm{即为}x+|x+2|\geq 6$

$\therefore \left\{\begin{array}{l}
x\leq - 2\\
x- x- 2\geq 6
\end{array}\right.$或$\left\{\begin{array}{l}
x>- 2\\
x+x+2\geq 6
\end{array}\right.$,即$x\geq 2$

$\therefore M=\{x|x\geq 2\}$\raisebox{-5pt}{{\ldots}{\ldots}}5分

@101MA{\textbar}S23\#2@(Ⅱ)

@答案@

@典型错误@

@解析@

由(Ⅰ)知$\mathrm{m}=2$,即$\mathrm{a}+\mathrm{b}=2$且$a,{\quad} b\in R^{+}$

\begin{align*}
\begin{array}{l}
\therefore \left(\frac{1}{a}+1\right)\left(\frac{1}{b}+1\right)=\left(\frac{a+b}{2a}+1\right)\\
=\left(\frac{b}{2a}+\frac{3}{2}\right)\left(\frac{a}{2b}+\frac{3}{2}\right)=\frac{5}{2}+\frac{3}{4}\left(\frac{b}{a}+\frac{a}{b}\right)\geq \frac{5}{2}+\frac{3}{4}\times 2\sqrt{\frac{b}{a}\times \frac{a}{b}}=4
\end{array}
\end{align*}

当且仅当$\mathrm{a}=\mathrm{b}=1$时,$\left(\frac{1}{a}+1\right)\left(\frac{1}{b}+1\right)$取得最小值4{\ldots}{\ldots}10分

@110MA{\textbar}S24@24

@答案@

@典型错误@

@解析@

@101MA{\textbar}S24\#1@\raisebox{-2.5pt}{(Ⅰ)}

@答案@

@典型错误@

@解析@

解:(Ⅰ)由已知$S_{n}=\frac{3}{2}a_{n}- \frac{1}{2}$\ding{172},

得$S_{n- 1}=\frac{3}{2}a_{n- 1}- \frac{1}{2},(n\geq 2)$\ding{173},

\ding{172}-\ding{173}得$a_{n}=\frac{3}{2}a_{n}- \frac{3}{2}a_{n- 1}$,即$a_{n}=3a_{n- 1}(n\geq 2)$

又$\mathrm{a}_{1}=1$,所以数列$\left\{a_{n}\right\}$是以1为首项, 3 为公比的等比数列,即$a_{n}=3^{n- 1}$\raisebox{-6pt}{,}{\ldots}{\ldots}5分

@101MA{\textbar}S24\#2@(Ⅱ)

@答案@

@典型错误@

@解析@

 由(Ⅰ)知$b_{n}=\frac{1}{n(n+1)}=\frac{1}{n}- \frac{1}{n+1}$

\begin{equation*}
\therefore T_{n}=\frac{1}{1}- \frac{1}{2}+\frac{1}{2}- \frac{1}{3}+......+\frac{1}{n}- \frac{1}{n+1}=1- \frac{1}{n+1}=\frac{n}{n+1}
\end{equation*}

$\therefore T_{n}=\frac{n}{n+1}${\ldots}{\ldots}10分

\textbf{\textit{H}}

\textbf{\textit{F}}

\textbf{\textit{E}}

\end{document}
